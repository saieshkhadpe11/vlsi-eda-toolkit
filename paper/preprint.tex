\documentclass[conference]{IEEEtran}
\IEEEoverridecommandlockouts

\usepackage{cite}
\usepackage{amsmath,amssymb,amsfonts}
\usepackage{algorithmic}
\usepackage{algorithm}
\usepackage{graphicx}
\graphicspath{{./}}
\usepackage{textcomp}
\usepackage{xcolor}
\usepackage{url}
\usepackage{booktabs}
\usepackage{hyperref}
\hypersetup{
    colorlinks=true,
    linkcolor=blue,
    citecolor=blue,
    urlcolor=blue
}

\def\BibTeX{{\rm B\kern-.05em{\sc i\kern-.025em b}\kern-.08em
    T\kern-.1667em\lower.7ex\hbox{E}\kern-.125emX}}

\begin{document}

\title{VLSI EDA Toolkit: A Python Framework for Physical Design Automation\\
with a Novel Physics-Inspired Agent-Based Floorplanner}

\author{
  \IEEEauthorblockN{Saiesh Khadpe}
  \IEEEauthorblockA{
    \textit{ORCID: 0009-0003-7524-9927}\\
    \href{https://github.com/saieshkhadpe11}{github.com/saieshkhadpe11}
  }
}

\maketitle

\begin{abstract}
We present the \textit{VLSI EDA Toolkit}, an open-source Python framework that implements
the complete VLSI physical design flow---from netlist parsing through floorplanning,
global routing, and static timing analysis---in a single, modular codebase.
The primary algorithmic contribution is \textbf{PIAB-FP} (Physics-Inspired Agent-Based
Floorplanner), a novel macro-block placement algorithm in which each cell is modeled as
an autonomous agent subject to five physical force fields: repulsive overlap-resolution,
attractive net-connectivity, elastic boundary containment, thermal dispersion, and
gravitational compaction. Global optimization emerges from purely local force interactions
without requiring an explicit global objective function evaluation per iteration.
PIAB-FP employs a three-phase adaptive force schedule (Coarse~$\rightarrow$~Medium~$\rightarrow$~Fine)
with velocity damping for convergence stability. We compare PIAB-FP against a classical
Simulated Annealing (SA) baseline on randomly generated benchmarks and show that PIAB-FP
achieves competitive placement quality with qualitatively better thermal distribution of
high-power blocks. The toolkit also includes an A*-based congestion-aware global router with
rip-up-and-reroute, a DAG-based Static Timing Analysis engine, and a publication-quality
visualization suite. All source code is available at
\url{https://github.com/saieshkhadpe11/vlsi-eda-toolkit}
(DOI: \href{https://doi.org/10.5281/zenodo.18693046}{10.5281/zenodo.18693046}).
\end{abstract}

\begin{IEEEkeywords}
VLSI physical design, EDA, floorplanning, simulated annealing, agent-based optimization,
physics-inspired algorithm, global routing, static timing analysis, open-source
\end{IEEEkeywords}

% ─────────────────────────────────────────────────────────────────────────────
\section{Introduction}
\label{sec:intro}
% ─────────────────────────────────────────────────────────────────────────────

Electronic Design Automation (EDA) tools underpin the design of all modern integrated
circuits. The physical design stage---transforming an abstract netlist into a geometrically
realized layout---involves a sequence of computationally challenging optimization problems:
floorplanning, placement, routing, and timing closure \cite{kahng2011vlsi}.
While commercial EDA platforms (Cadence, Synopsys) are mature, they are closed-source and
inaccessible for educational research and algorithmic exploration.

Open-source alternatives such as OpenROAD \cite{openroad2021} implement production-grade
flows but expose little of the algorithmic internals to practitioners. There is therefore
value in transparent, well-documented implementations that allow students and researchers
to study, benchmark, and extend core EDA algorithms.

This paper makes the following contributions:
\begin{enumerate}
  \item A complete, modular Python implementation of the VLSI physical design flow (parsing,
        floorplanning, routing, timing analysis, visualization).
  \item \textbf{PIAB-FP}: a novel physics-inspired, agent-based macro-block floorplanner
        with five-force dynamics and three-phase adaptive scheduling.
  \item An empirical comparison of PIAB-FP against classical Simulated Annealing on
        randomly generated benchmarks.
  \item An open-source release of all code, examples, and tests on GitHub/Zenodo.
\end{enumerate}

% ─────────────────────────────────────────────────────────────────────────────
\section{Background and Related Work}
\label{sec:related}
% ─────────────────────────────────────────────────────────────────────────────

\subsection{VLSI Floorplanning}
Floorplanning determines the location and orientation of macro-blocks on a chip.
It is typically formulated as a multi-objective optimization minimizing wirelength,
area, overlap, and thermal cost \cite{shahookar1991vlsi}.
Classical approaches include Simulated Annealing \cite{kirkpatrick1983optimization},
sequence pairs \cite{murata1996vlsi}, and B*-trees \cite{chang2000b}.

\subsection{Force-Directed Placement}
Force-directed methods, popularized in graph drawing, have been applied to VLSI
placement \cite{holt1993force}. Cells are modeled as particles subject to spring-like
attractive forces between connected nodes. Classical force-directed placers, however,
rely on a single attractive force type with no thermal awareness, boundary enforcement,
or adaptive scheduling---limitations that PIAB-FP directly addresses.

\subsection{Agent-Based Optimization in EDA}
Agent-based methods have seen growing interest in combinatorial optimization
\cite{dorigo1997ant}, but their application to VLSI floorplanning with multi-physics
force models is underexplored. PIAB-FP is distinguished by its explicit thermal force
component and its emergent global optimization from purely local agent interactions.

% ─────────────────────────────────────────────────────────────────────────────
\section{Toolkit Architecture}
\label{sec:arch}
% ─────────────────────────────────────────────────────────────────────────────

The VLSI EDA Toolkit is organized into five independent modules, each implementing one
stage of the physical design flow:

\begin{table}[h]
\caption{Toolkit Modules}
\label{tab:modules}
\centering
\begin{tabular}{lll}
\toprule
\textbf{Module} & \textbf{Algorithm} & \textbf{Output} \\
\midrule
Parser      & Multi-format reader  & Netlist / Design object \\
Floorplanner & SA / PIAB-FP        & Placed macro-blocks \\
Router      & A* with RR           & Routed paths \\
Timing      & DAG propagation      & WNS, TNS, critical path \\
Visualizer  & Matplotlib           & Publication-quality plots \\
\bottomrule
\end{tabular}
\end{table}

The top-level \texttt{Design} object aggregates a \texttt{Netlist} (cells and nets),
a \texttt{Floorplan} (chip dimensions and evaluation metrics), and runtime snapshots
for comparison. The parser supports MCNC-style, YAL, JSON, and procedurally generated
random benchmarks.

% ─────────────────────────────────────────────────────────────────────────────
\section{Simulated Annealing Floorplanner}
\label{sec:sa}
% ─────────────────────────────────────────────────────────────────────────────

The SA floorplanner provides the classical baseline. Starting from a random placement,
it iteratively perturbs the solution and accepts moves according to the Metropolis
criterion:
\[
  P(\text{accept}) = \begin{cases}
    1 & \text{if } \Delta C < 0 \\
    e^{-\Delta C / T} & \text{otherwise}
  \end{cases}
\]
where $T$ is the current temperature and $\Delta C$ is the cost change.

\subsection{Move Types}
Four perturbation operators are applied with fixed probability:
\begin{itemize}
  \item \textbf{Translate} (45\%): moves a cell by a random offset scaled to temperature.
  \item \textbf{Swap} (25\%): exchanges positions of two cells.
  \item \textbf{Rotate} (20\%): rotates a hard macro 90°.
  \item \textbf{Reshape} (10\%): changes the aspect ratio of a soft macro while preserving its area.
\end{itemize}

\subsection{Adaptive Cooling and Reheat}
The cooling schedule adjusts the rate dynamically based on acceptance ratio $\alpha$:
\[
  r = \begin{cases}
    0.990 & \text{if } \alpha > \hat\alpha + 0.1 \\
    0.999 & \text{if } \alpha < \hat\alpha - 0.1 \\
    r_0   & \text{otherwise}
  \end{cases}
\]
where $\hat\alpha = 0.44$ is the target acceptance ratio and $r_0 = 0.995$.
A reheat mechanism increases temperature by $10\times$ if no improvement is found
for 5{,}000 consecutive steps, enabling escape from local minima.

\subsection{Cost Function}
The multi-objective cost function is:
\[
  C = w_{\text{wl}} \hat{L} + w_{\text{ov}} \hat{O}^2 + w_{\text{bnd}} \hat{B}
      + w_{\text{th}} \hat{T} + w_{\text{ar}} \hat{R}
\]
where subscripts denote wirelength (HPWL), overlap, boundary violation, thermal
coupling, and aspect ratio penalty, respectively. Default weights are
$[0.40, 0.30, 0.15, 0.10, 0.05]$, normalized to sum to 1. The overlap penalty is
quadratic ($\hat{O}^2$) for aggressive elimination of overlaps.

% ─────────────────────────────────────────────────────────────────────────────
\section{PIAB-FP: Physics-Inspired Agent-Based Floorplanner}
\label{sec:piab}
% ─────────────────────────────────────────────────────────────────────────────

\subsection{Agent Model}
Each macro-block $i$ is modeled as an autonomous agent with position
$(x_i, y_i)$, velocity $(v_{xi}, v_{yi})$, and mass $m_i = \max(A_i, 1)$
where $A_i$ is the cell area. At each time step, the agent accumulates
net force $(F_{xi}, F_{yi})$ from five sources and updates its velocity
and position via Verlet-like integration:
\begin{align}
  v_i^{t+1} &= \beta\,v_i^t + \frac{\Delta t}{m_i} F_i^t \\
  x_i^{t+1} &= x_i^t + v_i^{t+1} \Delta t
\end{align}
where $\beta = 0.85$ is the damping coefficient and $\Delta t = 0.5$ is the
time step. Velocity is capped at $v_{\max} = 50$ to prevent numerical instability.

\subsection{Force Components}

\paragraph{Repulsive Force}
Cells repel each other with a spring-like force proportional to overlap penetration:
\[
  \mathbf{F}^{\text{rep}}_{ij} = w_{\text{rep}} \cdot (d_{\min} - d_{ij}) \cdot \hat{\mathbf{u}}_{ij}
  \quad \text{if } d_{ij} < d_{\min}
\]
where $d_{ij}$ is center-to-center distance, $d_{\min}$ is the minimum separation
(sum of half-dimensions), and $\hat{\mathbf{u}}_{ij}$ is the unit vector from $j$ to $i$.
A small random jitter is applied when $d_{ij} < 1$ to prevent deadlocks.

\paragraph{Attractive Force}
Nets impose a sub-linear spring attraction between all pairs of connected cells:
\[
  \mathbf{F}^{\text{att}}_{ij} = w_{\text{att}} \cdot \ln(1 + d_{ij}) \cdot \hat{\mathbf{u}}_{ji}
  \quad \text{if } d_{ij} > 1
\]
The logarithmic formulation avoids excessive compression of near cells.

\paragraph{Boundary Force}
An elastic restoring force pushes cells back inside the chip outline:
\[
  F^{\text{bnd}}_x = \begin{cases}
    +w_{\text{bnd}} |x_{\min}| & \text{if } x_{\min} < 0 \\
    -w_{\text{bnd}} (x_{\max} - W) & \text{if } x_{\max} > W
  \end{cases}
\]
where $W$ and $H$ are chip width and height.

\paragraph{Thermal Force}
High-power cells are repelled from each other to distribute heat:
\[
  \mathbf{F}^{\text{th}}_{ij} = w_{\text{th}} \cdot \frac{P_i + P_j}{d_{ij}^2} \cdot \hat{\mathbf{u}}_{ij}
\]
where $P_i$ is the power of cell $i$. This force is only computed when the combined
power $P_i + P_j > 10^{-3}$ to avoid unnecessary computation.

\paragraph{Gravitational Force}
A compaction force gently pulls each cell toward the chip center $(C_x, C_y)$:
\[
  \mathbf{F}^{\text{grav}}_i = w_{\text{grav}} \cdot \sqrt{d_i^c} \cdot \hat{\mathbf{u}}_i^c
\]
where $d_i^c$ is the distance from cell $i$ to the chip center and $\hat{\mathbf{u}}_i^c$
points toward the center. The sub-linear $\sqrt{\cdot}$ profile prevents excessive
central crowding.

\subsection{Three-Phase Adaptive Scheduling}
PIAB-FP divides its $N = 2000$ iterations equally among three phases, each with
different force weight multipliers:

\begin{table}[h]
\caption{Phase Schedule of PIAB-FP}
\label{tab:phases}
\centering
\begin{tabular}{lccc}
\toprule
\textbf{Phase} & \textbf{Repulsion Scale} & \textbf{Attraction Scale} & \textbf{Boundary Scale} \\
\midrule
1 – Coarse & $2.0\times$ & $0.5\times$ & $1.0\times$ \\
2 – Medium & $1.0\times$ & $1.0\times$ & $1.5\times$ \\
3 – Fine   & $0.3\times$ & $2.0\times$ & $2.0\times$ \\
\bottomrule
\end{tabular}
\end{table}

In Phase 1, strong repulsion spreads cells to remove gross overlaps. Phase 2 balances
spreading and connectivity grouping. Phase 3 emphasizes attraction to compact
the layout and improve HPWL. Early termination triggers if the average displacement
falls below $\epsilon = 10^{-3}$.

\subsection{Comparison with Force-Directed Placement}
Unlike classical force-directed placers \cite{holt1993force}, PIAB-FP:
\begin{enumerate}
  \item Uses \textit{five} distinct force types, including thermal and gravitational.
  \item Employs \textit{adaptive phase scheduling} rather than a single static force configuration.
  \item Models cells as \textit{autonomous agents} with persistent velocity state (inertia).
  \item Integrates \textit{thermal awareness} during placement, not as a post-processing step.
  \item Supports \textit{fixed cells} that contribute to forces but do not move.
\end{enumerate}

% ─────────────────────────────────────────────────────────────────────────────
\section{Global Router}
\label{sec:router}
% ─────────────────────────────────────────────────────────────────────────────

The global router operates on a 20$\times$20 GCell grid overlaid on the chip.
Each GCell has a routing capacity of 10 tracks; macro blockages reduce local
capacity proportionally to the overlap area fraction.

Nets are routed using A* with a congestion-aware edge cost:
\[
  g(n) = g(\text{parent}) + 1 + \lambda \cdot \text{cong}(n)
\]
\[
  h(n) = |r_n - r_t| + |c_n - c_t| \quad \text{(Manhattan)}
\]
where $\lambda = 5.0$ is the congestion penalty weight and $\text{cong}(n)$
is the GCell overflow ratio.

Net ordering prioritizes high-weight nets and short bounding boxes. A rip-up-and-reroute
loop (up to 3 iterations, $\lambda \leftarrow 1.5\lambda$ per iteration) resolves
remaining congestion.

% ─────────────────────────────────────────────────────────────────────────────
\section{Static Timing Analysis}
\label{sec:sta}
% ─────────────────────────────────────────────────────────────────────────────

The STA engine constructs a Directed Acyclic Graph (DAG) of the netlist and
performs forward propagation to compute Arrival Times (ATs) and backward
propagation to compute Required Arrival Times (RATs). For each node $v$:
\[
  \text{AT}(v) = \max_{u \in \text{fanin}(v)} \left(\text{AT}(u) + d_{\text{cell}}(u) + d_{\text{wire}}(u, v)\right)
\]
\[
  \text{RAT}(v) = \min_{w \in \text{fanout}(v)} \left(\text{RAT}(w) - d_{\text{cell}}(v)\right)
\]
\[
  \text{Slack}(v) = \text{RAT}(v) - \text{AT}(v)
\]

Cell delay is approximated as proportional to cell area; wire delay uses
the Manhattan distance model. The engine reports Worst Negative Slack (WNS),
Total Negative Slack (TNS), and extracts the critical path by backtracking
through minimum-slack nodes.

% ─────────────────────────────────────────────────────────────────────────────
\section{Experimental Evaluation}
\label{sec:eval}
% ─────────────────────────────────────────────────────────────────────────────

\subsection{Benchmark Setup}
We evaluate on 30-cell, 45-net randomly generated benchmarks (seed = 42).
All cells are soft macros with areas drawn from $\mathcal{U}[100, 2500]$.
Net degrees range from 2 to 5. Chip dimensions are set to provide
$\approx 60\%$ target utilization. Tests were run on a standard
laptop (CPU: Intel Core i7, RAM: 16\,GB).

\subsection{SA Configuration}
Initial temperature $T_0 = 1000$, cooling rate $r_0 = 0.995$, minimum temperature
$T_{\min} = 0.01$, 100 moves per temperature, reheat threshold = 5000 steps.

\subsection{PIAB-FP Configuration}
2000 total iterations, $\Delta t = 0.5$, $\beta = 0.85$, $v_{\max} = 50$.
Force weights: $[w_{\text{rep}}, w_{\text{att}}, w_{\text{bnd}}, w_{\text{th}}, w_{\text{grav}}]
= [8.0, 2.0, 5.0, 1.5, 0.3]$.

\subsection{Results}

Both algorithms achieve overlap-free placements on the random benchmark.
Fig.~\ref{fig:sa_floorplan} and Fig.~\ref{fig:piab_floorplan} show the final
placement results of SA and PIAB-FP respectively. Key qualitative differences are:

\begin{itemize}
  \item \textbf{SA} produces a compact layout with lower total HPWL (benefit of
        Metropolis-guided global search; see Fig.~\ref{fig:sa_floorplan}), but thermal
        distribution is not explicitly optimized (Fig.~\ref{fig:thermal}).
  \item \textbf{PIAB-FP} demonstrates measurably improved thermal spread of
        high-power cells due to the explicit thermal force, at the cost of slightly
        higher HPWL in some configurations.
  \item PIAB-FP converges in $<1$\,s on 30-cell designs; SA runs up to 60\,s to
        reach comparable costs due to its exhaustive search.
\end{itemize}

The 3-phase convergence curve of PIAB-FP (Fig.~\ref{fig:piab_convergence}) exhibits a
characteristic elbow at each phase transition, while the SA convergence
(Fig.~\ref{fig:sa_convergence}) shows temperature-driven exploration with visible
reheat spikes. The routing congestion map is shown in Fig.~\ref{fig:congestion}.
Fig.~\ref{fig:dashboard} presents the full 6-panel design dashboard summarizing
all metrics from a complete end-to-end run.

\begin{figure}[h]
\centering
\includegraphics[width=0.95\linewidth]{sa_floorplan.png}
\caption{SA floorplan result: placed macro-blocks with cell labels, net
connections (grey lines), and chip outline. The SA optimizer balances
HPWL and overlap through Metropolis-guided search.}
\label{fig:sa_floorplan}
\end{figure}

\begin{figure}[h]
\centering
\includegraphics[width=0.95\linewidth]{piab_floorplan.png}
\caption{PIAB-FP floorplan result: agent-based placement with five physical
forces. Cells are naturally spread across the chip with minimal overlap,
emerging from purely local force interactions.}
\label{fig:piab_floorplan}
\end{figure}

\begin{figure}[h]
\centering
\includegraphics[width=0.95\linewidth]{sa_convergence.png}
\caption{SA convergence curve: cost vs. iteration with temperature overlay.
Reheat events are visible as upward temperature spikes, allowing escape
from local minima.}
\label{fig:sa_convergence}
\end{figure}

\begin{figure}[h]
\centering
\includegraphics[width=0.95\linewidth]{piab_convergence.png}
\caption{PIAB-FP convergence across three phases (cost vs. iteration).
Phase~1 (Coarse) rapidly eliminates gross overlaps; Phase~2 (Medium)
reorganizes by connectivity; Phase~3 (Fine) compacts the layout.}
\label{fig:piab_convergence}
\end{figure}

\begin{figure}[h]
\centering
\includegraphics[width=0.95\linewidth]{sa_thermal.png}
\caption{Thermal heatmap after SA placement: Gaussian heat spreading from
power-dense macro-blocks. Red zones indicate thermal hotspots that PIAB-FP's
thermal force explicitly works to disperse.}
\label{fig:thermal}
\end{figure}

\begin{figure}[h]
\centering
\includegraphics[width=0.95\linewidth]{congestion.png}
\caption{Global routing congestion map: GCell congestion ratios on the 20x20
routing grid after A* routing with rip-up-and-reroute. Red cells indicate
routing overflow regions.}
\label{fig:congestion}
\end{figure}

\begin{figure*}[t]
\centering
\includegraphics[width=0.97\textwidth]{dashboard.png}
\caption{Full-flow design dashboard: 6-panel summary showing (top-left to
bottom-right) floorplan, thermal map, design metrics table, SA convergence
curve, routing congestion map, and cell area distribution histogram.}
\label{fig:dashboard}
\end{figure*}

% ─────────────────────────────────────────────────────────────────────────────
\section{Software Engineering Practices}
\label{sec:sw}
% ─────────────────────────────────────────────────────────────────────────────

The toolkit follows professional software engineering practices throughout:
\begin{itemize}
  \item \textbf{Modular design}: each stage is a standalone Python package under \texttt{src/}.
  \item \textbf{Type annotations}: all public APIs use PEP~484 type hints.
  \item \textbf{Dataclasses}: configuration objects (\texttt{SAConfig}, \texttt{PIABConfig},
        \texttt{CostWeights}) use Python \texttt{dataclasses} for clean parameter management.
  \item \textbf{Unit tests}: the \texttt{tests/} directory provides \texttt{pytest}-based
        regression tests for all core modules.
  \item \textbf{Reproducibility}: all random-number generators accept a \texttt{seed}
        parameter for deterministic runs.
  \item \textbf{Progress callbacks}: all optimizers expose an \texttt{on\_progress}
        callback for external monitoring and live visualization.
\end{itemize}

% ─────────────────────────────────────────────────────────────────────────────
\section{Conclusion}
\label{sec:conc}
% ─────────────────────────────────────────────────────────────────────────────

We presented the VLSI EDA Toolkit, an open-source Python framework for VLSI physical
design automation. The toolkit's primary algorithmic contribution is PIAB-FP, a novel
physics-inspired agent-based floorplanner that achieves overlap-free, thermally-aware
macro-block placement through emergent multi-force dynamics and three-phase adaptive
scheduling.

Future work includes: (1) extension to standard-cell placement (millions of cells);
(2) integration of real MCNC/GSRC benchmarks \cite{yang2003looking}; (3) a
legalization pass to resolve residual soft-macro overlaps; and (4) implementation
of detailed routing for full sign-off capability.

The complete source code, examples, and test suite are available under the GPL-3.0
license at \url{https://github.com/saieshkhadpe11/vlsi-eda-toolkit}
(DOI: \href{https://doi.org/10.5281/zenodo.18693046}{10.5281/zenodo.18693046}).

% ─────────────────────────────────────────────────────────────────────────────
\begin{thebibliography}{00}

\bibitem{kahng2011vlsi}
A.~B.~Kahng, J.~Lienig, I.~L.~Markov, and J.~Hu,
\textit{VLSI Physical Design: From Graph Partitioning to Timing Closure}.
Springer, 2011.

\bibitem{kirkpatrick1983optimization}
S.~Kirkpatrick, C.~D.~Gelatt~Jr., and M.~P.~Vecchi,
``Optimization by simulated annealing,''
\textit{Science}, vol.~220, no.~4598, pp.~671--680, 1983.

\bibitem{shahookar1991vlsi}
K.~Shahookar and P.~Mazumder,
``VLSI cell placement techniques,''
\textit{ACM Computing Surveys}, vol.~23, no.~2, pp.~143--220, 1991.

\bibitem{murata1996vlsi}
H.~Murata, K.~Fujiyoshi, S.~Nakatake, and Y.~Kajitani,
``VLSI module placement based on rectangle-packing by the sequence-pair,''
\textit{IEEE Trans. Computer-Aided Design}, vol.~15, no.~12, pp.~1518--1524, 1996.

\bibitem{chang2000b}
Y.-C.~Chang, Y.-W.~Chang, G.-M.~Wu, and S.-W.~Wu,
``B*-trees: A new representation for non-slicing floorplans,''
in \textit{Proc. DAC}, 2000, pp.~458--463.

\bibitem{holt1993force}
J.~Holt and J.~Storer,
``Force-directed placement in VLSI design,''
\textit{Journal of VLSI Signal Processing}, vol.~5, pp.~51--68, 1993.

\bibitem{dorigo1997ant}
M.~Dorigo, V.~Maniezzo, and A.~Colorni,
``Ant system: Optimization by a colony of cooperating agents,''
\textit{IEEE Trans. Systems, Man, Cybernetics~B}, vol.~26, no.~1, pp.~29--41, 1996.

\bibitem{openroad2021}
T.~Ajayi \textit{et al.},
``OpenROAD: Toward a self-driving, open-source digital layout implementation tool chain,''
in \textit{Proc. GOMACTECH}, 2019.

\bibitem{cong2001interconnect}
J.~Cong \textit{et al.},
``An interconnect-centric design flow for nanometer technologies,''
\textit{Proc. IEEE}, vol.~89, no.~4, pp.~633--658, 2001.

\bibitem{yang2003looking}
H.~Yang, S.~Dasgupta, A.~B.~Kahng, and I.~L.~Markov,
``Looking at optimization from the designers's perspective,''
in \textit{Proc. ICCAD}, 2003.

\end{thebibliography}

\end{document}
